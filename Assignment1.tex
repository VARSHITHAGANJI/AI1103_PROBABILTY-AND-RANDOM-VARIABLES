\documentclass[journal,12pt,twocolumn]{IEEEtran}
\usepackage[utf8]{inputenc}
\usepackage{amsmath}
\usepackage{amssymb}

\title{\textbf{PROBABILITY AND RANDOM VARIABLES (AI1103)}\\ Assignment 1}
\author{GANJI VARSHITHA-AI20BTECH11009}


\begin{document}

\maketitle

\section*{Question}
\textbf{Assigned Problem 6.9}
\\
If A and B are two events such that $A\subset B$ and $P(B)\neq 0$, then which of the following is correct?
\begin{enumerate}
   \item P($\frac{A}{B}$) = $\frac{P(B)}{P(A)}$
   \item P($\frac{A}{B}$) $<$ P(A)
   \item P($\frac{A}{B}$) $\geq$ P(A)
   \item None of these
\end{enumerate}
\section*{Solution}
We know that A is the subset of B. \\
$\Rightarrow$ Every element of A is an element of B.\\
\begin{equation}
  \therefore A\cap B = A
  \label{eq1}
\end{equation}
We know that 
\begin{equation}
 \begin{split}
 P\Big(\frac{A}{B}\Big)&= \frac{P(A\cap B)}{P(B)}  \\ 
 &=\frac{P(A)}{P(B)}
 \label{eq2}
\end{split}
\end{equation}
Given $0<P(B)\leq1$
\\ $\Rightarrow \frac{1}{P(B)} \geq 1$\\

\textnormal{By multiplying with P(A) on both sides of the inequality, we get\\
\\
$\frac{P(A)}{P(B)}\geq$ }{\normalsize{P(A)}}\\

Using \ref{eq2}, we have 
$$P \Big (\frac{A}{B} \Big) \geq P(A)$$
Therefore, option 3 is correct.


\end{document}
